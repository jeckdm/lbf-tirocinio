\documentclass[../../main.tex]{subfiles}

\graphicspath{{\subfix{../../immagini/}}}

\begin{document}
    Il dataset utilizzato è composto da circa $350.000$ esempi di URL etichettati come legittimi o di phishing; nello specifico, circa $310.000$ degli URL sono etichettati come legittimi, i $40.000$ rimanenti come phishing.

    Il dataset è stato costruito unendo i dati ricavati da tre fonti differenti: 
    \begin{itemize}
        \item $15.000$ esempi per entrambe le etichette sono stati ricavati dal dataset fornito in \cite{article},
        \item $290.000$ esempi di URL legittimi sono stati estratti dalla web directory \cite{botw},
        \item i rimanenti esempi sono stati ricavati dal dataset fornito dal Machine Learning Lab dell'Università di Trieste \cite{machinelearninglab}.
    \end{itemize}

    Nei nostri esperimenti l'insieme delle chiavi $\mathcal{K}$ comprende tutti gli URL di phishing, i rimanenti URL legittimi faranno invece parte dell'insieme delle non-chiavi $\mathcal{U}$.
\end{document}