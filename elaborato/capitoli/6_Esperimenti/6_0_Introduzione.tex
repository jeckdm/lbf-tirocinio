\documentclass[../../main.tex]{subfiles}

\graphicspath{{\subfix{../../immagini/}}}

\begin{document}
    Scopo di questo capitolo è di fornire una panoramica sugli strumenti utilizzati negli esperimenti, i cui risultati vengono riportati nel capitolo successivo. 
    
    In generale, lo scopo dei nostri esperimenti è quello di confrontare le performance delle due tipologie di filtri appresi (introdotte nel capitolo \ref{chap:filtriAppresi}) che differiscono per la tipologia di classificatore utilizzata, questo al fine di comprendere se un classificatore più complesso, e quindi più oneroso in termini di spazio, possa portare vantaggi rispetto ad un classificatore meno complesso e di conseguenza più piccolo, ma presumibilmente meno efficiente. Nello specifico, la struttura del classificatore complesso è equivalente a quella della GRU presentata in \cite{ma2020}, mentre il classificatore più semplice è un percettrone multistrato. Le strutture delle due reti vengono presentate nel dettaglio rispettivamente nei paragrafi \ref{sec:strutturaRNN} e \ref{sec:strutturaPercettrone}.

    Gli esperimenti effettuati si dividono in due categorie: nella prima rientrano tutti gli esperimenti volti a valutare, e confrontare, le performance dei soli classificatori. I risultati di questa prima parte vengono riportati nel paragrafo \ref{sec:valutazioneClassificatori}. Nella seconda categoria, invece, rientrano gli esperimenti utili a confrontare le performance di filtri appresi in cui vengono `inseriti' i classificatori analizzati del paragrafo precedente. I risultati di questa seconda parte vengono riportati nel paragrafo \ref{sec:confrontoFiltri}.

    Il problema di classificazione considerato è un problema di classificazione binaria che consiste nel riconoscere URL legittimi e URL di phishing, il dataset utilizzato negli esperimenti comprende circa 350.000 elementi, la struttura di questo dataset viene presentata più nel dettaglio nel prossimo paragrafo. 
\end{document}