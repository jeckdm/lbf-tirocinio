\documentclass[../../main.tex]{subfiles}

\graphicspath{{\subfix{../../immagini/}}}

\begin{document}
    Seguendo lo stesso ragionamento già fatto nel paragrafo precedente per l'LBF, l'equazione che quantifica il tasso di falsi positivi di un SLBF su query $\mathcal{Q}$ composte da elementi $x$ estratti dall'insieme $\mathcal{X} - \mathcal{K}$ secondo una distribuzione $\mathcal{D}$ è la seguente: 
    \begin{equation}
        f = f_{b_0}\left(\underset{x \sim \mathcal{D}}{\mathbb{P}}(g(x) \geq \tau) + (1 - \underset{x \sim \mathcal{D}}{\mathbb{P}}(g(x) \geq \tau))f_b\right),
        \label{eqn:SLBFFalsiPositivi}
    \end{equation}
    qui la notazione è equivalente a quella utilizzata nel paragrafo precedente: $g$ indica il modello, $\tau$ la soglia, $f_b$ ed $f_{b0}$ indicano rispettivamente il tasso di falsi positivi del filtro di backup e del filtro iniziale.

    L'equazione può essere ricavata pensando a come avviene il processo di controllo dell'appartenenza per un $x \notin \mathcal{K}$: l'elemento ha una probabilità pari a $f_{b0}$ di essere erroneamente etichettato come positivo dal filtro, dopodiché il processo si svolge come già descritto per l'LBF. Per ricavare $f$ quindi basta semplicemente moltiplicare \eqref{eqn:LBFFalsiPositivi} per $f_{b0}$.
\end{document}