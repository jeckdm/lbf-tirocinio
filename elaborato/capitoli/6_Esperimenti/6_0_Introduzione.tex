\documentclass[../../main.tex]{subfiles}

\graphicspath{{\subfix{../../immagini/}}}

\begin{document}
    Lo scopo di questo capitolo è fornire una panoramica sugli strumenti utilizzati negli esperimenti, i cui risultati vengono riportati nel capitolo successivo. 
    
    \section{Introduzione}
    Gli esperimenti presentati mirano a confrontare le performance delle tipologie di filtri di Bloom appresi (introdotte nel Capitolo \ref{chap:filtriAppresi}) a cui vengono applicate diverse tipologie di classificatore, questo al fine di comprendere se un classificatore più complesso, e quindi più oneroso in termini di spazio, possa portare vantaggi rispetto ad un classificatore meno complesso e di conseguenza più piccolo, ma presumibilmente meno accurato. Nello specifico, la struttura del classificatore complesso è equivalente a quella della GRU presentata in \cite{ma2020}, mentre il classificatore più semplice è un percettrone multistrato. Le strutture delle due reti vengono presentate nel dettaglio rispettivamente nei Paragrafi \ref{sec:strutturaRNN} e \ref{sec:strutturaPercettrone}.

    Gli esperimenti effettuati si dividono in due categorie: nella prima rientrano tutti gli esperimenti volti a valutare e confrontare, le performance dei soli classificatori. Questa prima parte di esperimenti è utile a fornire un'idea dell'efficacia dei classificatori nel problema del riconoscimento degli URL. I risultati di questa prima parte vengono riportati nel Paragrafo \ref{sec:valutazioneClassificatori}. Nella seconda categoria, invece, rientrano gli esperimenti utili a confrontare le performance di filtri appresi in cui vengono `inseriti' i classificatori analizzati del paragrafo precedente. I risultati di questa seconda parte vengono riportati nel Paragrafo \ref{sec:confrontoFiltri}.

    Il problema considerato è un problema di classificazione binaria che consiste nel riconoscimento di URL legittimi e URL di phishing, utilizzando un dataset la cui struttura è presentata nel prossimo paragrafo.
\end{document}