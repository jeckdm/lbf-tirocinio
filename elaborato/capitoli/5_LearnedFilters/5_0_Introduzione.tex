\documentclass[../../main.tex]{subfiles}

\graphicspath{{\subfix{../../immagini/}}}

\begin{document}
    Nonostante i filtri di Bloom, introdotti nel Capitolo \ref{chap:FiltriBloom}, siano strutture per definizione efficienti in termini di spazio, molto spesso in applicazioni pratiche, in cui il numero di elementi da salvare nel filtro è molto alto, lo spazio richiesto per il filtro potrebbe comunque risultare limitante. Per questo motivo esistono moltissime varianti della struttura del filtro originale, tutte con l'obiettivo di migliorare il più possibile l'efficienza della struttura; le due varianti utilizzate negli esperimenti descritti nel Capitolo \ref{chap:Esperimenti} prendono il nome di \textit{learned Bloom filter}, o filtro di Bloom appreso \cite{kraska2018case}, e \textit{sandwiched Bloom filter} \cite{10.5555/3326943.3326986}. L'obiettivo di questo capitolo è fornire un'introduzione alle strutture e ai vantaggi delle due tipologie di filtro sopra citate.
\end{document}