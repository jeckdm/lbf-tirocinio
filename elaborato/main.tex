\documentclass{report}

% Margini
\usepackage[utf8]{inputenc}
\usepackage[margin=1in]{geometry}
\usepackage[titletoc,title]{appendix}

% Subfiles
\usepackage{subfiles}

% Immagini
\usepackage{graphicx}
\usepackage{subcaption}
\usepackage[export]{adjustbox}
\usepackage{float}
\usepackage{wrapfig}
\graphicspath{immagini/}

% Bibliografia
\usepackage[backend = biber, citestyle = alphabetic]{biblatex}
\addbibresource{bibliografia/bibliografia.bib}

% Formule, definizioni
\usepackage{amsmath}
\usepackage{amsthm}
\usepackage{amsfonts}
\usepackage{bm}
\usepackage{mathtools}
\usepackage{nccmath}
\newtheorem{dfn}{Definizione}[chapter]

% Spacing paragrafi
\usepackage{parskip}

% Pseudocodice
\usepackage{algorithm}
\usepackage{algpseudocode}
\algnewcommand\algorithmicforeach{\textbf{for each}}
\algdef{S}[FOR]{ForEach}[1]{\algorithmicforeach\ #1\ \algorithmicdo}

\begin{document}

\tableofcontents

\chapter{Introduzione}

\chapter{Aspetti Teorici}
\section{Definizioni di Base}

\chapter{Filtri di Bloom}

\chapter{Apprendimento automatico}
\subfile{capitoli/4_ApprendimentoAutomatico/4_0_ApprendimentoAutomatico}
\section{Classificazione e regressione}
\subfile{capitoli/4_ApprendimentoAutomatico/4_1_Classificazione_Regressione.tex}
\section{Reti neurali profonde} % Introduzione sugli aspetti più importanti del deep learning (Quali?) + focus su Reti FF e Reti Ricorrenti
\subfile{capitoli/4_ApprendimentoAutomatico/4_2_0_RetiProfonde.tex}
\subsection{Percettrone multistrato}
\subfile{capitoli/4_ApprendimentoAutomatico/4_2_1_PercettroneMultiStrato.tex}
\subsection{Reti Neurali Ricorrenti}
\subfile{capitoli/4_ApprendimentoAutomatico/4_2_2_RetiRicorrenti.tex}
\subsection{Overfitting e Underfitting}
\subfile{capitoli/4_ApprendimentoAutomatico/4_2_3_SceltaIperparametri.tex}

\chapter{Filtri di Bloom appresi}
    \section{Learned Bloom Filters (LBF)}
    \section{Sandwitched Learned Bloom Filters (SLBF)}

% Titolo da cambiare
\chapter{Esperimenti}

% Spiego scopo degli esperimenti: rapporto tra classificatori e filtri, capire se è veramente necessario una rete complessa come la RNN o se una rete FFNN è sufficiente per migliorare le prestazioni
\section{Valutazione del ruolo dei classificatori nei filtri di Bloom appresi}

% Illustro i dataset utilizzati, eventualmente confrontando con quelli utilizzati nell'articolo ed eventualmente spiego la suddivisione del dataset usata per il testing dei filtri spiegandone la motivazione
\section{Dati}

% Spiegazione teorica delle tecinche utilizzate per l'analisi delle prestazioni dei classificatori e l'ottimizzazione dei parametri
\section{Metriche di valutazione}
\subsection{Convalida incrociata} % Nome?
\subsection{Selezione del modello} % Gridsearch e nested CV

% Illustro la struttura RNN e FFNN utilizzate
\section{Struttura dei classificatori utilizzati}
\subsection{Percettrone multistrato}
\subsection{Rete ricorrente}


% Spiegazione delle codifiche utilizzate, con eventuale spiegazione del perchè non posso utilizzare per entrambe le reti la stessa codifica
% + eventuale codifica binaria ancora da testare
% Spiegazione delle tecniche utilizzate per la serializzazione dei modelli ed il perchè della scelte: pickle vs funzioni di libreria, menziono inoltre le differenze ottenute sopratutto utilizzando pytorch nelle dimensioni del modello prima e dopo l'addestramento
\section{Implementazione}

\chapter{Risultati}

% Esperimenti al fine di valutare la bontà della rete ricorrente: voglio ottimizzare la rete ricorrente sul mio dataset e poi valutare le prestazioni sui filtri di bloom
% Esperimenti per valutare prestazioni della rete feedforward sul dataset + ottimizzazione dei parametri e del modello tramite funzioni di keras
\section{Valutazione dei classificatori}


% Confronto risultati riguardanti le prestazioni dei classificatori e le relative differenze nelle prestazioni dei filtri appresi 
% Traggo le conclusioni degli esperimenti: è necessario utilizzare una rete complessa come quella ricorrente o posso utilizzare anche reti più "semplici" come una feed forward?
\section{Valutazione sull'impatto dei classificatori nei filtri di Bloom appresi}

\chapter{Conclusioni}


\printbibliography

\end{document}