\documentclass[../../main.tex]{subfiles}

\graphicspath{{\subfix{../../immagini/}}}

\begin{document}
    Lo scopo principale di questo elaborato è stato lo studio dei filtri di Bloom appresi, concentrandosi nello specifico sulle due tipologie che prendono il nome di filtro di Bloom appreso, o learned Bloom filter, e sandwiched learned Bloom filter. In particolare, l'analisi ha riguardato il ruolo del classificatore in tale struttura, e come esso potesse incidere sullo spazio occupato dal filtro. I classificatori analizzati sono stati due, entrambi basati su reti neurali: un percettrone multistrato, e una rete ricorrente basata su una GRU, con struttura equivalente a quella presentata in \cite{ma2020}. 
    
    Il problema di classificazione affrontato riguarda il riconoscimento di URL; nello specifico, l'obiettivo è stato costruire un classificatore in grado di discernere, in modo sufficientemente accurato, tra URL legittimi e di phishing. I risultati presentati hanno evidenziato, per il problema considerato, delle prestazioni migliori da parte del percettrone. Conseguentemente, anche le taglie dei filtri appresi ottenuti utilizzando tale modello sono risultate più basse rispetto a quelle ottenute sfruttando la rete ricorrente. Infine, un confronto tra le taglie delle due tipologie di filtro di Bloom appreso ha confermato, come già mostrato in altri lavori, una maggiore efficienza del sandwiched learned Bloom filter a parità di tasso di falsi positivi.

    Tuttavia, è anche emerso che le varianti apprese del filtro risultano più lente di 1-2 ordini di grandezza rispetto al filtro classico, ciò potrebbe risultare più o meno limitante a seconda dell'ambito in cui la struttura viene utilizzata: in contesti dove lo spazio è una risorsa preziosa, questo compromesso risulta comunque accettabile. 
    
    Futuri lavori potrebbero quindi concentrarsi su diversi aspetti, primo fra questi l'analisi di filtri appresi con tipologie di classificatore più semplici e non basate su reti neurali, come i classificatori lineari, al fine di cercare un miglior compromesso tra spazio risparmiato e tempi d'accesso. Altro aspetto interessante su cui concentrarsi potrebbe essere la ricerca di criteri ottimali, attualmente assenti, per la scelta della soglia $\tau$. In generale, le potenzialità di queste strutture dati vanno ancora comprese pienamente.
\end{document}