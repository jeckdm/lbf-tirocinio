\documentclass[../../main.tex]{subfiles}

\graphicspath{{\subfix{../../immagini/}}}

\begin{document}
    Viene dimostrato in \cite{10.5555/3326943.3326986} che la dimensione ottima $n_b^*$, in termini di minimizzazione del tasso di falsi positivi globale, del filtro di Bloom di backup può essere calcolata tramite la seguente equazione: 
    \begin{equation}
        n_b^* = m_b \cdot \log_{\alpha}\left(\frac{F_p}{(1-F_p)(\frac{m}{m_b} - 1)}\right),
        \label{eqn:SLBFGrandezzaOttima}
    \end{equation}
    questa può essere ricavata come segue: assumiamo di avere a disposizione un totale di $n = bm$ bit da dividere tra filtro iniziale $n_{b_0} = b_1 m$ e il filtro di backup $n_{b} = b_2 m$, e che i due filtri rispettino la condizione di ottimalità per $k$, sotto queste condizioni possiamo riscrivere l'equazione \ref{eqn:SLBFFalsiPositivi} come: 
    \[f = \alpha^{b_1}\left(F_p + (1 - F_p)\alpha^{b_2/F_n}\right), \]
    essendo $\alpha$, $F_p$, $F_n$ e $b$ delle costanti è possibile trovare il valore ottimo per $n_b$ calcolando la derivata e ponendola uguale a 0.
    
    Partendo dalla seguente equazione: 
    \[f = \alpha^{b_1}F_p + (1 - F_p)\alpha^{b_1(1 - 1/F_n)} \alpha^{b/F_n}, \]
    in cui $b_2$ è stato sostituito con $b - b_1$, calcoliamo la derivata rispetto a $b_1$ e la poniamo uguale a 0: 
    \begin{flalign*}
        &F_p (\ln\alpha)\alpha^{b_1} + (1 - F_p) \left(1 - \frac{1}{F_n}\right)\alpha^{b/F_n}(\ln\alpha)\alpha^{b_1(1 - 1/F_n)} = 0,\\
        &\frac{F_p}{(1-F_p)(\frac{1}{F_n} - 1)} = \alpha^{(b-b_1)/F_n} =  \alpha^{b_2/F_n},\\
        &b_2 = F_n \log_\alpha \left(\frac{F_p}{(1-F_p)(\frac{1}{F_n} - 1)}\right).
    \end{flalign*}
    Ricordando che $F_n$ può essere calcolato come rapporto tra chiavi etichettate erroneamente come non chiavi e chiavi totali $m_b/m$, sostituendo ritroviamo l'equazione \ref{eqn:SLBFGrandezzaOttima}:
    \[\underbrace{b_2 m}_{n_b^*} = m_b \log_\alpha \left(\frac{F_p}{(1-F_p)(\frac{m}{m_b} - 1)}\right).\]
    Una volta trovato il valore ottimo per la dimensione del filtro di backup è facile ricavare il relativo valore del tasso di falsi positivi $f_b$, ricordando infatti che vale la relazione $f_b = \alpha^{n_b^*/m_b}$ sostituisco $n_b^*$: 
    \begin{equation}
        f_b = \frac{F_p}{(1 - F_p)(\frac{m}{m_b} - 1)}.
        \label{eqn:SLBFfBOttimo}
    \end{equation}
    Assumendo di avere un filtro di backup con dimensione ottima $n_b^*$, il cui tasso di falsi positivi può essere di conseguenza calcolato tramite l'equazione \ref{eqn:SLBFfBOttimo}, possiamo ricavare delle equazioni per il tasso di falsi positivi del filtro iniziale e la sua grandezza.

    È necessario prima di tutto sostituire il valore di $f_b$ nell'equazione \ref{eqn:SLBFFalsiPositivi}, per poi risolvere in funzione di $f_{b0}$: 

    \begin{flalign*}
        &f = f_{b0}\left(F_p + (1 - F_p) \cdot \left(\frac{F_p}{(1 - F_p)(\frac{1}{F_n} - 1)}\right)\right),\\
        &f = f_{b0}\left(F_p + \left(\frac{F_p}{(\frac{1}{F_n} - 1)}\right)\right),\\
        &f = f_{b0}\left(F_p \left(1 - \frac{1}{F_n}\right)\right),\\
    \end{flalign*}
    ottenendo (sostituendo $F_n$):
    \begin{equation}
        f_{b0} = \frac{f}{F_p} \left(1 - \frac{m_b}{m}\right),
    \end{equation}
    da cui, infine, possiamo ricavare il valore ottimo per $n_{b0}^*$: 
    \begin{equation}
        n_{b0}^* = m \log_\alpha\left(\frac{f}{F_p} \left(1 - \frac{m_b}{m}\right)\right).
    \end{equation}

    È importante notare come l'equazione \ref{eqn:SLBFGrandezzaOttima} riveli che la grandezza ottima per il filtro di backup è una costante, e non dipende nè dal numero $b$ di bit a disposizione nè dal tasso di falsi positivi desiderato; ne consegue che, assumendo di avere un filtro di backup di taglia $n_b^*$, all'aumentare del numero di bit a disposizione l'azione migliore è quella di aumentare la taglia del filtro iniziale.
    
\end{document}