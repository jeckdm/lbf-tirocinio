\documentclass[../../main.tex]{subfiles}

\graphicspath{{\subfix{../../immagini/}}}

\begin{document}
    Una volta introdotta l'Equazione \eqref{eqn:LBFFalsiPositivi} è possibile ricavare un limite inferiore sulla grandezza della struttura una volta definito il tasso di falsi positivi $\epsilon$ desiderato.

    Il procedimento per ricavare tale limite è simile a quello già presentato nel Capitolo \ref{chap:FiltriBloom}, adattandolo però alle nuove componenti presenti nell'LBF.

    Partiamo supponendo di avere un LBF composto da: 
    \begin{itemize}
        \item un insieme di chiavi $\mathcal{K}$ con cardinalità $m = |\mathcal{K}|$,
        \item un modello $g$, addestrato su un dataset $\mathcal{T}$ come descritto nell'introduzione di questo paragrafo,
        \item un filtro di backup $B$.
    \end{itemize}
    Considerando il classificatore, ogni chiave avrà una probabilità $F_n$ di essere erroneamente etichettata come non-chiave, generando un falso negativo. Viceversa, ogni non-chiave avrà una probabilità $F_p$ di essere etichetta come chiave, generando un falso positivo. 
    
    Definiamo  la grandezza del filtro di backup $n = m \cdot b$, tale filtro dovrà contenere solamente i falsi negativi generati dal classificatore, quindi un totale di $F_n \cdot m$ elementi.

    Fissiamo ora un tasso di falsi positivi $\epsilon$ desiderato ed imponiamo che il tasso di falsi positivi $f$ della struttura sia minore di tale valore: 
    \[f < \epsilon.\]
    Utilizzando le notazioni introdotte e sostituendo ad $f$ il suo valore, come definito in \eqref{eqn:LBFFalsiPositivi}, possiamo riscrivere la precedente disequazione come
    \begin{equation}
        F_p+ (1 - F_p)f_b < \epsilon,
    \end{equation}
    assumendo che il filtro di backup abbia un numero $k$ di funzioni di hash ottimo, per quanto detto nel Capitolo \ref{chap:FiltriBloom} vale $f_b = \alpha^{(bm)/(F_nm)} = \alpha^{b/F_n}$, con $\alpha \approx 0.6185$. Sostituendo il valore di $f_b$ nella disequazione e risolvendo rispetto a $b$: 
    \begin{flalign*}
        &F_p + (1 - F_p)\alpha^{b/F_n} < \epsilon,\\            
        &\alpha^{b/F_n} < \frac{\epsilon - F_p}{1 - F_p},\\
        &b \geq F_n \log_\alpha\left(\frac{\epsilon - F_p}{1 - F_p}\right).
    \end{flalign*}
    Dato che il tasso di falsi negativi $F_n$ può essere calcolato come rapporto tra chiavi etichettate erroneamente come non chiavi e chiavi totali $m_b/m$, otteniamo la seguente disequazione: 
    \begin{equation}
        \underbrace{b \cdot m}_n \geq m_b \log_\alpha\left(\frac{\epsilon - F_p}{1 - F_p}\right).
    \end{equation}
    Seguendo un ragionamento simile è facile ricavare, fissato $\epsilon$, il tasso di falsi positivi $f_b$ del filtro di backup: 
    \[F_p + (1 - F_p)f_b = \epsilon,\]
    ottenendo: 
    \begin{equation}
        f_b = \frac{\epsilon - F_p}{1 - F_p}.
        \label{eqn:LBFBackupFalsePositive}
    \end{equation}
    Importante notare che affinché \eqref{eqn:LBFBackupFalsePositive} non generi un tasso negativo è necessario che valga $\epsilon > F_p$.
\end{document}