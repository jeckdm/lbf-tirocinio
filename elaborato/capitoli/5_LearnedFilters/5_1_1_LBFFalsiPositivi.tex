\documentclass[../../main.tex]{subfiles}

\graphicspath{{\subfix{../../immagini/}}}

\begin{document}
    
    Una delle problematiche degli LBF deriva proprio dalla sua caratteristica principale, ovvero l'utilizzo di un classificatore: ricordiamo infatti che nei filtri di Bloom è dimostrabile che in applicazioni reali la probabilità di falsi positivi è molto concentrata attorno al suo valore atteso (equazione \ref{eqn:bloomfpr}), questa proprietà nei filtri appresi può essere garantita solo sotto determinate assunzioni.

    A causa del classificatore infatti il tasso di falsi positivi in un LBF è dipendente dalla distribuzione delle query $\mathcal{Q}$ che gli vengono presentate. Per dimostrare a livello pratico la precedente affermazione è utile riportare l'esempio presentato in \cite{10.5555/3326943.3326986}: supponiamo di avere un insieme universo di elementi compresi tra $[0, 10000)$ e di costruire un LBF che salva 50 elementi estratti casualmente dall'intervallo $[200,300]$ e 50 elementi estratti casualmente dagli intervalli restanti. Supponiamo ora che il classificatore $g$, dopo l'addestramento, ritorni un valore $g(x) \approx \frac{1}{2}$ per gli elementi nell'intervallo $[200,300]$, ed un valore vicino a 0 per tutti gli altri. Assumendo una soglia $\tau = 0.4$, è facile notare che presentando al filtro query con elementi estratti uniformemente dall'intervallo $[200,300]$ la probabilità di ottenere falsi positivi sarà molto più alta rispetto a query in cui gli elementi vengono estratti uniformemente dall'intervallo $[0, 10000)$.

    Nonostante la dipendenza dalla distribuzione dei dati presentati, è comunque possibile definire il tasso di falsi positivi di un LBF: dato un classificatore addestrato $g$, una soglia $\tau$ ed una query di elementi estratti secondo la distribuzione $\mathcal{D}$ dall'insieme $\mathcal{X} - \mathcal{K}$, con $\mathcal{X}$ insieme universo, il tasso di falsi positivi $f$ del filtro appreso è espresso dalla seguente equazione:
    \begin{equation}
        f = \underset{x \sim \mathcal{D}}{\mathbb{P}}(g(x) \geq \tau) + (1 - \underset{x \sim \mathcal{D}}{\mathbb{P}}(g(x) \geq \tau))f_b.
        \label{eqn:LBFFalsiPositivi}
    \end{equation} 
    La notazione $f_b$ rappresenta il tasso di falsi positivi del filtro di backup, tale quantità è una variabile aleatoria, tuttavia, per quanto detto nel Capitolo \ref{chap:FiltriBloom} questa risulta nella pratica concentrata attorno al suo valore atteso.

    Infine, supponendo di avere un test set $\mathcal{G}$ tale per cui $\mathcal{G} \cap \mathcal{K} = \emptyset$ ed assumendo che questo abbia una distribuzione equivalente a quella delle query $\mathcal{Q}$ che verranno poi presentate al filtro, è possibile determinare in modo empirico il tasso di falsi positivi della struttura semplicemente come rapporto tra i falsi positivi in $\mathcal{G}$ e la sua cardinalità $|\mathcal{G}|$. Viene dimostrato in \cite{10.5555/3326943.3326986} che il tasso empirico di $\mathcal{G}$ calcolato sotto queste assunzioni risulta molto simile al tasso empirico di falsi positivi delle query $\mathcal{Q}$. 
    
    Quest'ultimo risultato suggerisce inoltre un possibile approccio per la scelta della soglia $\tau$: dato che, indipendentemente da $\tau$, il tasso di falsi positivi empirico calcolato su $\mathcal{G}$ risulta molto simile a quello delle query $\mathcal{Q}$, un possibile approccio è testare un insieme discreto di valori di $\tau$ e scegliere quello che porta alla creazione del filtro migliore.
\end{document}