\documentclass[../../main.tex]{subfiles}

\graphicspath{{\subfix{../../immagini/}}}

\begin{document}
    \subsubsection{Proporzione training-testing}
    Il primo esperimento viene effettuato sulle tre tipologie di GRU descritte, ed il fine è quello di esaminare le prestazioni del classificatore al variare della proporzione training set, testing set. 
    
    Questo esperimento è utile ad evidenziare eventuali bias negativi dovuti ad un sottodimensionamento dell'insieme d'addestramento; intuitivamente, essendo il dataset utilizzato relativamente grande, non ci aspettiamo cambiamenti di performance significative nelle proporzioni testate: il training set dovrebbe avere in ogni caso una grandezza sufficiente per permettere al modello di apprendere il problema al meglio delle sue potenzialità.

    Vengono testate tre proporzioni training-testing: $\frac{1}{2}$ $\frac{1}{2}$, $\frac{2}{3}$ $\frac{1}{3}$ e $\frac{4}{5}$ $\frac{1}{5}$, che corrispondono rispettivamente ad una 2 fold, 3 fold e 5 fold cross validation.

    \begin{table}[H]
        \centering
        \begin{tabular}{lccc}
            \toprule
            {} &                      \textbf{16 Dimensioni} & \textbf{8 Dimensioni} & \textbf{4 Dimensioni} \\
            \midrule
            \textbf{F1-score }      &      $0.720 \pm 0.038$ & $0.657 \pm 0.008$ & $0.563 \pm 0.136$\\
            \textbf{Recupero   }    &      $0.720 \pm 0.096$ & $0.647 \pm 0.012$ & $0.485 \pm 0.183$\\
            \textbf{Precisione}     &      $0.731 \pm 0.022$ & $0.668 \pm 0.002$ & $0.727 \pm 0.006$\\
            \textbf{Accuratezza }   &      $0.932 \pm 0.004$ & $0.917 \pm 0.001$ & $0.914 \pm 0.015$\\
            \bottomrule
        \end{tabular}
        \caption{Risultati medi di una 2 fold cross validation, calcolati sulla classe dei phishing.}
        \label{tab:2foldCV}
    \end{table}

    \begin{table}[H]
        \centering
        \begin{tabular}{lccc}
            \toprule
            {} &                      \textbf{16 Dimensioni} & \textbf{8 Dimensioni} & \textbf{4 Dimensioni} \\
            \midrule
            \textbf{F1-score }      &      $0.694 \pm 0.029$ & $0.659 \pm 0.061$ & $0.592 \pm 0.040$\\
            \textbf{Recupero   }    &      $0.659 \pm 0.077$ & $0.625 \pm 0.088$ & $0.545 \pm 0.079$\\
            \textbf{Precisione}     &      $0.744 \pm 0.044$ & $0.702 \pm 0.025$ & $0.660 \pm 0.028$\\
            \textbf{Accuratezza }   &      $0.929 \pm 0.004$ & $0.921 \pm 0.009$ & $0.908 \pm 0.001$\\
            \bottomrule
        \end{tabular}
        \caption{Risultati medi di una 3 fold cross validation, calcolati sulla classe dei phishing.}
        \label{tab:3foldCV}
    \end{table}

    \begin{table}[H]
        \centering
        \begin{tabular}{lccc}
            \toprule
            {} &                      \textbf{16 Dimensioni} & \textbf{8 Dimensioni} & \textbf{4 Dimensioni} \\
            \midrule
            \textbf{F1-score }      &      $0.745 \pm 0.011$ & $0.666 \pm 0.064$ & $0.569 \pm 0.051$\\
            \textbf{Recupero   }    &      $0.791 \pm 0.034$ & $0.655 \pm 0.114$ & $0.526 \pm 0.103$\\
            \textbf{Precisione}     &      $0.707 \pm 0.045$ & $0.689 \pm 0.011$ & $0.641 \pm 0.055$\\
            \textbf{Accuratezza }      &      $0.933 \pm 0.006$ & $0.921 \pm 0.009$ & $0.903 \pm 0.005$\\
            \bottomrule
        \end{tabular}
        \caption{Risultati medi di una 5 fold cross validation, calcolati sulla classe dei phishing.}
        \label{tab:5foldCV}
    \end{table}

    Le tabelle \ref{tab:2foldCV}, \ref{tab:3foldCV}, \ref{tab:5foldCV} riportano i risultati al variare delle proporzioni. Come ci aspettavamo, le prestazioni non sembrano essere influenzate dalla variazione della grandezza del training set: la tabella \ref{tab:2foldCV}, ad esempio, presenta un F1-score sensibilmente maggiore rispetto alla \ref{tab:3foldCV} per la GRU a 16 dimensioni, mentre nell'ultima tabella la stessa metrica questo valore risale.

    Dato che le prestazioni non sembrano essere influenzate dalla grandezza del training set, la scelta è di mantenere una proporzione $\frac{2}{3}$ $\frac{1}{3}$.

    \subsubsection{Sbilanciamento del dataset}
    Come riportato nel paragrafo \ref{sec:dataset}, il nostro dataset contiene un numero significativamente superiore di esempi etichettati come URL legittimi, nello specifico per ogni URL di phishing sono presenti circa sette URL legittimi.

    L'idea di questo esperimento è quindi quella di verificare se una variazione della proporzione legittimi:phishing possa portare ad un aumento delle performance del classificatore. Queste variazione viene attuata eliminando  casualmente URL legittimi, fino ad ottenere la proporzione desiderata, questa tecnica prende il nome di random undersampling. Anche in questo caso l'esperimento viene effettuato solamente sulle tre GRU.

    Le tabelle \ref{tab:5a1Undersampling}, \ref{tab:3a1Undersampling}, \ref{tab:2a1Undersampling} e \ref{tab:1a1Undersampling} mostrano i risultati di un 3-fold cross validation sulle proporzioni riportate.

    Confrontando questi risultati con quelli della tabella \ref{tab:3foldCV}, si notano dei miglioramenti significativi in quasi tutte le proporzioni testate: l'F1-score sembra in generale migliore lavorando su un dataset meno sbilanciato. È interessante notare però che, all'aumentare del numero di URL legittimi scartati, tale miglioramento diventi sempre minore, fino ad arrivare alla tabella \ref{tab:1a1Undersampling} in cui le performance in termini di F1-score sono peggiori rispetto alla tabella \ref{tab:3foldCV}.

    Questo progressivo peggioramento potrebbe essere giustificato dalla progressiva diminuzione della grandezza del dataset: eliminando URL legittimi per bilanciare il dataset, infatti, contemporaneamente si perdono molti esempi che sarebbero stati utili al modello per apprendere meglio il problema.

    Questa supposizione viene confermata anche analizzando precisione e recupero: maggiori sono gli URL legittimi eliminati,e maggiore è il recupero, a scapito però della precisione. Ciò indica, da un lato, un miglioramento nelle capacità di riconoscimento degli URL di phishing (recupero alto, quindi basso numero di falsi negativi), ma contemporaneamente un notevole peggioramento nelle capacità di riconoscimento degli URL legittimi (precisione bassa, quindi alto numero di falsi positivi).

    Per i motivi appena spiegati, la scelta è di mantenere una proporzione 5:1, che sembra essere quella che garantisce l'F1-score migliore, mantenendo comunque un buon bilanciamento tra recupero e precisione.


    \begin{table}[H]
        \centering
        \begin{tabular}{lccc}
            \toprule
            {} &                      \textbf{16 Dimensioni} & \textbf{8 Dimensioni} & \textbf{4 Dimensioni} \\
            \midrule
            \textbf{F1-Score }      &      $0.733 \pm 0.010$ & $0.751 \pm 0.005$ & $0.677 \pm 0.035$\\
            \textbf{Recupero   }    &      $0.834 \pm 0.026$ & $0.790 \pm 0.015$ & $0.711 \pm 0.063$\\
            \textbf{Precisione}     &      $0.654 \pm 0.011$ & $0.717 \pm 0.017$ & $0.649 \pm 0.046$\\
            \textbf{Accuratezza }   &      $0.925 \pm 0.003$ & $0.935 \pm 0.002$ & $0.916 \pm 0.009$\\
            \bottomrule
        \end{tabular}
        \caption{Proporzioni legittimi phishing 5:1. Risultati medi di una 3 fold cross validation, calcolati sulla classe dei phishing.}
        \label{tab:5a1Undersampling}
    \end{table}

    \begin{table}[H]
        \centering
        \begin{tabular}{lccc}
            \toprule
            {} &                      \textbf{16 Dimensioni} & \textbf{8 Dimensioni} & \textbf{4 Dimensioni} \\
            \midrule
            \textbf{F1-Score }      &      $0.734 \pm 0.012$ & $0.703 \pm 0.037$ & $0.706 \pm 0.027$\\
            \textbf{Recupero   }    &      $0.872 \pm 0.004$ & $0.902 \pm 0.029$ & $0.884 \pm 0.022$\\
            \textbf{Precisione}     &      $0.633 \pm 0.019$ & $0.580 \pm 0.059$ & $0.590 \pm 0.047$\\
            \textbf{Accuratezza }   &      $0.922 \pm 0.005$ & $0.905 \pm 0.019$ & $0.908 \pm 0.013$\\
            \bottomrule
        \end{tabular}     
        \caption{Proporzioni legittimi phishing 3:1. Risultati medi di una 3 fold cross validation, calcolati sulla classe dei phishing.}   
        \label{tab:3a1Undersampling}
    \end{table}

    \begin{table}[H]
        \centering
        \begin{tabular}{lccc}
            \toprule
            {} &                      \textbf{16 Dimensioni} & \textbf{8 Dimensioni} & \textbf{4 Dimensioni} \\
            \midrule
            \textbf{F1-Score }      &      $0.746 \pm 0.021$ & $0.698 \pm 0.018$ & $0.658 \pm 0.019$\\
            \textbf{Recupero   }    &      $0.911 \pm 0.010$ & $0.906 \pm 0.014$ & $0.928 \pm 0.004$\\
            \textbf{Precisione}     &      $0.632 \pm 0.032$ & $0.568 \pm 0.030$ & $0.510 \pm 0.024$\\
            \textbf{Accuratezza }   &      $0.923 \pm 0.009$ & $0.903 \pm 0.010$ & $0.880 \pm 0.011$\\
            \bottomrule
        \end{tabular}     
        \caption{Proporzioni legittimi phishing 2:1. Risultati medi di una 3 fold cross validation, calcolati sulla classe dei phishing.} 
        \label{tab:2a1Undersampling}  
    \end{table}

    \begin{table}[H]
        \centering
        \begin{tabular}{lccc}
            \toprule
            {} &                      \textbf{16 Dimensioni} & \textbf{8 Dimensioni} & \textbf{4 Dimensioni} \\
            \midrule
            \textbf{F1-Score }      &      $0.612 \pm 0.003$ & $0.632 \pm 0.010$ & $0.642 \pm 0.043$\\
            \textbf{Recupero   }    &      $0.986 \pm 0.004$ & $0.972 \pm 0.005$ & $0.957 \pm 0.021$\\
            \textbf{Precisione}     &      $0.444 \pm 0.004$ & $0.468 \pm 0.013$ & $0.486 \pm 0.053$\\
            \textbf{Accuratezza }   &      $0.846 \pm 0.002$ & $0.860 \pm 0.007$ & $0.866 \pm 0.028$\\
            \bottomrule
        \end{tabular}     
        \caption{Proporzioni legittimi phishing 1:1. Risultati medi di una 3 fold cross validation, calcolati sulla classe dei phishing.}  
        \label{tab:1a1Undersampling} 
    \end{table}

    \subsubsection{Performance percettrone multistrato}
    Andiamo ora a valutare le prestazioni del percettrone, questo avrà una struttura coerente con quella descritta nel paragrafo \ref{sec:strutturaPercettrone}. Il rapporto training-testing set utilizzato e lo sbilanciamento del dataset rimangono uguali a quelli scelti nei due esperimenti precedenti, le performance verranno quindi valutate tramite una 3-fold cross validation sul dataset con una proporzione legittimi:phishing pari a 5:1.

    Vengono testate in questo caso tre modelli diversi, differenti per il numero di neuroni dello strato interno. In questo primo esperimento le tre configurazioni testate sono state scelte arbitrariamente, negli esperimenti successivi il numero di neuroni verrà scelto cercando di massimizzare le performance mantenendo uno spazio occupato simile a quello occupato dalle GRU. Anche il learning rate viene scelto arbitrariamente, in questo caso viene fissato a $0.001$.

    \begin{table}[H]
        \centering
        \begin{tabular}{lccc}
            \toprule
            {} &                      \multicolumn{3}{c}{\textbf{(Numero neuroni, Learning rate)}}\\
            {} &                      \textbf{(64, 0.001)} & \textbf{(16, 0.001)} & \textbf{(8, 0.001)} \\
            \midrule
            \textbf{Spazio (Byte)}  &      $21762$ & $5628$ & $2940$\\
            \midrule
            \textbf{F1-score }      &      $0.846 \pm 0.002$ & $0.812 \pm 0.002$ & $0.789 \pm 0.003$\\
            \textbf{Recupero   }    &      $0.842 \pm 0.030$ & $0.830 \pm 0.008$ & $0.813 \pm 0.015$\\
            \textbf{Precisione}     &      $0.853 \pm 0.025$ & $0.794 \pm 0.003$ & $0.767 \pm 0.014$\\
            \textbf{Accuratezza}    &      $0.962 \pm 0.001$ & $0.952 \pm 0.001$ & $0.946 \pm 0.001$\\
            \bottomrule            
        \end{tabular}
        \caption{Risultati della model evaluation sul percettrone multistrato, effettuata con un 3-fold cross selection. Il dataset ha una proporzione URL legittimi URL phishing di 5 a 1.}
        \label{tab:FFNNModelSelection}
    \end{table}

    \begin{table}[H]
        \centering
        \begin{tabular}{lccc}
            \toprule
            {} &                      \textbf{16 Dimensioni} & \textbf{8 Dimensioni} & \textbf{4 Dimensioni} \\
            \midrule
            \textbf{Spazio (Byte)}  &      $9452$ & $6390$ & $5515$\\
            \midrule               
            \textbf{F1-Score }      &      $0.733 \pm 0.010$ & $0.751 \pm 0.005$ & $0.677 \pm 0.035$\\
            \textbf{Recupero   }    &      $0.834 \pm 0.026$ & $0.790 \pm 0.015$ & $0.711 \pm 0.063$\\
            \textbf{Precisione}     &      $0.654 \pm 0.011$ & $0.717 \pm 0.017$ & $0.649 \pm 0.046$\\
            \textbf{Accuratezza }   &      $0.925 \pm 0.003$ & $0.935 \pm 0.002$ & $0.916 \pm 0.009$\\
            \bottomrule
        \end{tabular}
        \caption{Risultati della model evaluation sulla GRU, effettuata con un 3-fold cross selection. Il dataset ha una proporzione URL legittimi URL phishing di 5 a 1.}
        \label{tab:GRUModelSelection}
    \end{table}

    La tabella \ref{tab:FFNNModelSelection} riporta i risultati della model evaluation effettuata sulle tre configurazioni. La tabella \ref{tab:GRUModelSelection}, invece, riprende i risultati della tabella \ref{tab:5a1Undersampling}, aggiungendo però anche le informazioni sullo spazio occupato da ogni GRU.

    È facile notare come tutti i modelli di percettrone abbiano performance migliori rispetto alle GRU. Inoltre, come era intuibile, all'aumentare del numero di neuroni le performance del percettrone migliorano. Se il problema fosse solamente quello di risolvere il problema di classificazione, il modello a 64 neuroni sarebbe il migliore tra le tre configurazioni. Tuttavia, il nostro obiettivo è inserire questi classificatori in un filtro appreso, di conseguenza lo spazio occupato è importante tanto quanto le prestazioni. Il modello a 64 neuroni risulta quindi troppo grande rispetto alla prestazioni offerte.
    
    L'obiettivo è ora trovare, tramite una model selection, la configurazione di parametri migliore per un percettrone, cercando di mantenere delle dimensioni che non siano di troppo superiori a quella della GRU a 16 dimensioni. Dalla tabella \ref{tab:FFNNModelSelection}, infatti, si nota che il miglioramento delle performance passando dalla configurazione a 16 neuroni a quella a 64 non è così significativo da giustificare una dimensione circa quattro volte superiore.

    \subsubsection{Model selection}

    La model selection viene effettuata tramite una nested cross validation con k=3 sia per ciclo interno che per il ciclo esterno. La griglia di parametri testata è la seguente:
    \[
    \begin{matrix*}[l]
        \text{numero neuroni}: & [8, 16, 20, 25, 30],\\
        \text{learning rate}: & [0.0001, 0.001, 0.01, 0.1].
    \end{matrix*}
    \]
    Nel ciclo interno le combinazioni di parametri vengono create grazie all'algoritmo GridSearch.

    Per i motivi presentati nei paragrafi precedenti, l'obiettivo della model selection è massimizzare l'F1-score. Essendo il nostro dataset lievemente sbilanciato l'accuratezza potrebbe non risultare una metrica adatta a quantificare la bontà del modello.
    

    La tabella \ref{tab:modelSelection} presenta i risultati della model selection. 
    \begin{table}[H]
        \centering                  
        \begin{tabular}{lccc}
            \toprule
            {}  &   \textbf{Percettrone (Model selection)}\\
            \midrule
            \textbf{F1-score }      &    $0.835 \pm 0.002$ \\
            \textbf{Recupero   }    &    $0.809 \pm 0.010$ \\
            \textbf{Precisione }    &    $0.863 \pm 0.011$ \\
            \textbf{Accuratezza }   &    $0.958 \pm 0.001$ \\
            \midrule
            {} & \textbf{(Numero neuroni, Learning Rate)}\\
            \midrule
            \textbf{Fold 1} &   (30, 0.001)\\ 
            \textbf{Fold 2} &   (30, 0.001)\\
            \textbf{Fold 3} &   (30, 0.01)\\
            \bottomrule
        \end{tabular}
        \caption{}
        \label{tab:modelSelection}
    \end{table}


    
\end{document}