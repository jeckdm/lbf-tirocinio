\documentclass[../../main.tex]{subfiles}

\graphicspath{{\subfix{../../immagini/}}}

\begin{document}
    Come accade in un filtro di Bloom classico, anche l'obiettivo di un LBF è quello di salvare in modo efficiente in termini di spazio un dato insieme, in questo capitolo chiamato $\mathcal{K}$, di elementi, o chiavi, e di fornire la possibilità di controllare l'appartenenza di un generico elemento $x$ a tale insieme.

    La differenza rispetto al filtro base si trova nell'utilizzo di un classificatore per aiutare nel processo di controllo dell'appartenenza di un elemento all'insieme. Di fatto, il problema dell'appartenenza viene considerato come un problema di classificazione binaria in cui gli elementi appartenenti al filtro possiedono l'etichetta 1, mentre i rimanenti elementi possiedono etichetta 0.

    Più formalmente, l'obiettivo è quello di apprendere un modello $g$ in grado di etichettare correttamente la maggior parte degli elementi appartenenti all'insieme; il classificatore viene quindi addestrato su un training set $\mathcal{T}$ definito come unione degli insiemi $\mathcal{K}$, insieme delle chiavi, ed $\mathcal{U}$, insieme delle non-chiavi:
    \begin{equation}
        \mathcal{T} = \{(x_i, y_i = 1) | x_i \in \mathcal{K}\} \cup \{(x_i, y_i = 0) | x_i \in \mathcal{U}\}.
    \end{equation}
    Essendo in un problema di classificazione binaria, $g$ avrà una funzione logistica come funzione d'attivazione e verrà addestrato minimizzando la seguente funzione di perdita $L$, chiamata log-loss:
    \begin{equation}
        L = \sum_{(x,y) \in \mathcal{T}}\left(y \log g(x) + (1 - y) \log(1 - g(x))\right).
        \label{eqn:logloss}
    \end{equation}
    Per come il modello $g$ è stato definito, il valore ritornato da $g(x)$ rappresenta la probabilità che l'elemento $x$ appartenga al filtro, si rende di conseguenza necessaria l'introduzione di una soglia $\tau$, un elemento verrà quindi giudicato come appartenente al filtro se $g(x) > \tau$.
    
    Utilizzando solamente il classificatore però nasce la possibilità di ottenere dei falsi negativi: elementi $x \in \mathcal{K}$ per cui $g(x)$ risulta inferiore alla soglia, questo viola una delle caratteristiche fondamentali del filtro di Bloom, ovvero l'assenza di falsi negativi. Per eliminare i falsi negativi prodotti dal classificatore viene introdotto un filtro di backup, questo è semplicemente un filtro di Bloom incaricato di salvare tutti gli elementi appartenenti all'insieme $\mathcal{K}_{\tau}^- = \{x \in \mathcal{K} | g(x) < \tau\}$, ovvero tutti gli elementi ingiustamente etichettati come non-chiavi.

    Infine, è utile per riassumere riportare la definizione di LBF data in \cite{10.5555/3326943.3326986}: 

    ``Un LBF $(g, \tau, B)$, definito su un insieme di chiavi $\mathcal{K}$ ed un insieme di non-chiavi $\mathcal{U}$ è composto da una funzione $g : \mathcal{X} \rightarrow [0,1]$ legata ad una soglia $\tau$, dove $\mathcal{X}$ è l'insieme universo che contiene chiavi e non chiavi, e da un filtro di Bloom $B$, chiamato filtro di backup, il cui compito è salvare tutti gli elementi dell'insieme $\mathcal{K}_{\tau}^- = \{x \in \mathcal{K} | g(x) < \tau\}$, ovvero l'insieme di falsi negativi generati da $g$. Data un qualsiasi elemento $y$, l'LBF ritornerà $y \in \mathcal{K}$ se $g(y) > \tau$, o se $g(y) < \tau$ ed il filtro di backup ritorna $y \in \mathcal{K}$. In tutti gli altri casi l'LBF ritornerà $y \notin \mathcal{K}$.''

    La figura \ref{fig:confrontoInserimento} mette a confronto la procedura di inizializzazione per le due tipologie di filtro presentate, nello specifico viene riportata nella figura \ref{fig:BFInizializzazione} l'inizializzazione per un filtro di Bloom, mentre in figura \ref{fig:LBFInizializzazione} quella di un LBF.

    \begin{figure}[H]
        \centering
        \includegraphics[width=\textwidth]{immagini/5_1/BFInizializzazione.png}
        \caption{Procedura di inizializzazione di un filtro di Bloom: tutti gli elementi dell'insieme $\mathcal{K}$ vengono passati attraverso le funzioni di hash $k$, i bit presenti nelle posizioni corrispondenti ai risultati delle funzioni di hash vengono posti ad 1.}
        \label{fig:BFInizializzazione}
    \end{figure}

    \begin{figure}[H]
        \centering
        \includegraphics[width=\textwidth/2]{immagini/5_1/LBFInizializzazione.png}
        \caption{Procedura di inizializzazione di un LBF: tutti gli elementi dell'insieme $\mathcal{K}$ vengono passati attraverso il modello addestrato sul dataset $\mathcal{T}$, se un elemento viene etichettato come negativo viene inserito nel filtro di backup.}
        \label{fig:LBFInizializzazione}
        \caption{}
        \label{fig:confrontoInserimento}
    \end{figure}
\end{document}