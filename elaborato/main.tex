\documentclass{article}

% Margini
\usepackage[utf8]{inputenc}
\usepackage[margin=1in]{geometry}
\usepackage[titletoc,title]{appendix}

\begin{document}

\section{Introduzione}

\section{Aspetti Teorici}
\subsection{Filtri di Bloom}
\subsubsection{Filtri di Bloom appresi}
\subsection{Deep Learning} % Introduzione sugli aspetti più importanti del deep learning (Quali?) + focus su Reti FF e Reti Ricorrenti
\subsubsection{Reti Neurali Feed Forward}
\subsubsection{Reti Neurali Ricorrenti}

% Titolo da cambiare
\section{Introduzione Esperimenti}

% Spiego scopo degli esperimenti: capire se è veramente necessario una rete complessa come la RNN o se una rete FFNN è sufficiente per migliorare le prestazioni
\subsection{Classificatori e Filtri di Bloom appresi}

% Spiegazione struttura RNN e FFNN utilizzate
\subsection{Struttura dei classificatori utilizzati}
\subsubsection{Rete ricorrente}
\subsubsection{Rete feed forward}

% Spiegazione teorica delle tecinche utilizzate per l'analisi delle prestazioni dei classificatori e l'ottimizzazione dei parametri
\subsection{Tecniche di validazione e ottimizzazione dei parametri}
\subsubsection{Cross Validation}
\subsubsection{Model selection}

% Spiegazione delle codifiche utilizzate, con eventuale spiegazione del perchè non posso utilizzare per entrambe le reti la stessa codifica
\subsection{Codifiche utilizzate}
\subsubsection{Codifica Rete Ricorrente}
\subsubsection{Codifica Count Vectorizer}
\subsubsection{Discussioni sulle differenze tra le codifiche}

% Illustro i dataset utilizzati, eventualmente confrontando con quelli utilizzati nell'articolo ed eventualmente spiego la suddivisione del dataset usata per il testing dei filtri spiegandone la motivazione
\subsection{Raccolta dei dati}

% Spiegazione delle tecniche utilizzate per la serializzazione dei modelli ed il perchè della scelte: pickle vs funzioni di libreria, menziono inoltre le differenze ottenute sopratutto utilizzando pytorch nelle dimensioni del modello prima e dopo l'addestramento
\subsection{Serializzazione dei modelli}

\section{Esperimenti}

% Esperimenti al fine di valutare la bontà della rete ricorrente: voglio ottimizzare la rete ricorrente sul mio dataset e poi valutare le prestazioni sui filtri di bloom
\subsection{Analisi delle prestazioni della Rete Ricorrente}
\subsubsection{Dimensioni degli insiemi di addestramento e testing}
\subsubsection{Analisi delle prestazioni variando lo sbilanciamento del dataset}
\subsubsection{Analisi del bias introdotto dalla codifica utilizzata}

% Esperimenti per valutare prestazioni della rete feedforward sul dataset + ottimizzazione dei parametri e del modello tramite funzioni di keras
\subsection{Analisi delle prestazioni e ottimizzazione della Rete feed forward}
\subsubsection{Risultati preliminari}
\subsubsection{Ottimizzazione dei parametri del modello} % Cross validation e Model selection
\subsubsection{Analisi delle prestazioni con EarlyStopping e BestModel}

% Confronto risultati riguardanti le prestazioni dei classificatori e le relative differenze nelle prestazioni dei filtri appresi 
\subsection{Confronto delle prestazioni dei classificatori}
\subsection{Analisi e confronto dei classificatori applicati ai Filtri appresi}
% Traggo le conclusioni degli esperimenti: è necessario utilizzare una rete complessa come quella ricorrente o posso utilizzare anche reti più "semplici" come una feed forward?
\subsubsection{Discussioni su bilanciamento tra spazio e prestazioni}

\section{Conclusioni}

\end{document}