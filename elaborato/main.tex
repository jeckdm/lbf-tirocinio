\documentclass[4apaper,11pt]{report}
\usepackage[italian]{babel}

% Margini
\usepackage[utf8]{inputenc}
\usepackage[titletoc,title]{appendix}

\usepackage[table]{xcolor}

% Subfiles
\usepackage{subfiles}

% Immagini
\usepackage{graphicx}
\usepackage[labelformat=simple]{subcaption}
\usepackage[export]{adjustbox}
\usepackage{float}
\usepackage{wrapfig}
\usepackage{booktabs}
\usepackage{multirow}
\renewcommand\thesubfigure{(\alph{subfigure})}
\graphicspath{immagini/}

% Bibliografia
\usepackage[backend = biber, citestyle = numeric]{biblatex}
\addbibresource{bibliografia/bibliografia.bib}

% Formule, definizioni
\usepackage{amsmath}
\usepackage{amsthm}
\usepackage{amsfonts}
\usepackage{bm}
\usepackage{mathtools}
\usepackage{nccmath}

% Spacing paragrafi
\usepackage{parskip}
\usepackage[noabbrev, italian, capitalize]{cleveref}

% Pseudocodice
\usepackage{algorithm}
\usepackage{algpseudocode}
\floatname{algorithm}{Algoritmo}
\algnewcommand\algorithmicforeach{\textbf{for each}}
\algdef{S}[FOR]{ForEach}[1]{\algorithmicforeach\ #1\ \algorithmicdo}

% Tabelle
\usepackage{arydshln}

\newcommand{\crefrangeconjunction}{--}
\newcommand{\crefrangepreconjunction}{}
\begin{document}

\tableofcontents

\chapter*{Introduzione}
\subfile{capitoli/1_Introduzione/Introduzione.tex}

\chapter{Filtri di Bloom}
\label{chap:FiltriBloom}
\subfile{capitoli/3_FiltriBloom/3_FiltriBloom.tex}
\chapter{Apprendimento automatico}
\label{chap:ApprendimentoAutomatico}
\subfile{capitoli/4_ApprendimentoAutomatico/4_0_ApprendimentoAutomatico}
\section{Classificazione e regressione}
\subfile{capitoli/4_ApprendimentoAutomatico/4_1_Classificazione_Regressione.tex}
\section{Reti neurali profonde} % Introduzione sugli aspetti più importanti del deep learning (Quali?) + focus su Reti FF e Reti Ricorrenti
\subfile{capitoli/4_ApprendimentoAutomatico/4_2_0_RetiProfonde.tex}
\subsection{Percettrone multistrato}
\subfile{capitoli/4_ApprendimentoAutomatico/4_2_1_PercettroneMultiStrato.tex}
\subsection{Reti Neurali Ricorrenti}
\label{sec:retiRicorrenti}
\subfile{capitoli/4_ApprendimentoAutomatico/4_2_2_RetiRicorrenti.tex}
\subsection{Overfitting e Underfitting}
\label{sec:overfitting}
\subfile{capitoli/4_ApprendimentoAutomatico/4_2_3_SceltaIperparametri.tex}


\chapter{Filtri di Bloom appresi}
\label{chap:filtriAppresi}
\subfile{capitoli/5_LearnedFilters/5_0_Introduzione.tex}
\section{Learned Bloom Filters (LBF)}
\subfile{capitoli/5_LearnedFilters/5_1_0_LBF.tex}
\subsection{Probabilità di falsi positivi}
\label{sec:falseProbLBF}
\subfile{capitoli/5_LearnedFilters/5_1_1_LBFFalsiPositivi.tex}
\subsection{Dimensioni della struttura}
\subfile{capitoli/5_LearnedFilters/5_1_2_LBFGrandezza.tex}
\section{Sandwitched Learned Bloom Filters (SLBF)}
\subfile{capitoli/5_LearnedFilters/5_2_0_SLBF.tex}
\subsection{Probabilità di falsi positivi}
\subfile{capitoli/5_LearnedFilters/5_2_1_SLBFFalsiPositivi.tex}
\subsection{Dimensioni della struttura}
\subfile{capitoli/5_LearnedFilters/5_2_2_SLBFGrandezza.tex}

% Titolo da cambiare
\chapter{Esperimenti}
\label{chap:Esperimenti}
\subfile{capitoli/6_Esperimenti/6_0_Introduzione.tex}
\label{sec:Esperimenti}

% Spiego scopo degli esperimenti: rapporto tra classificatori e filtri, capire se è veramente necessario una rete complessa come la RNN o se una rete FFNN è sufficiente per migliorare le prestazioni

% Illustro i dataset utilizzati, eventualmente confrontando con quelli utilizzati nell'articolo ed eventualmente spiego la suddivisione del dataset usata per il testing dei filtri spiegandone la motivazione
\section{Dataset utilizzato}
\label{sec:dataset}
\subfile{capitoli/6_Esperimenti/6_2_Dati.tex}

% Spiegazione teorica delle tecinche utilizzate per l'analisi delle prestazioni dei classificatori e l'ottimizzazione dei parametri
\section{Tecniche di valutazione e selezione dei modelli}
\label{sec:metrichevalutazione}
\subfile{capitoli/6_Esperimenti/6_3_0_Introduzione.tex}
\subsection{Metriche di valutazione}
\subfile{capitoli/6_Esperimenti/6_3_1_MetricheValutazione.tex}
\subsection{Valutazione del modello} % Nome?
\subfile{capitoli/6_Esperimenti/6_3_2_ValutazioneModello.tex}
\subsection{Selezione del modello} % Gridsearch e nested CV
\subfile{capitoli/6_Esperimenti/6_3_3_SelezioneModello.tex}

% Illustro la struttura RNN e FFNN utilizzate
\section{Struttura dei classificatori utilizzati}
\subfile{capitoli/6_Esperimenti/6_4_0_Introduzione.tex}
\subsection{Percettrone multistrato}
\label{sec:strutturaPercettrone}
\subfile{capitoli/6_Esperimenti/6_4_1_StrutturaPercettrone.tex}
\subsection{Rete ricorrente}
\label{sec:strutturaRNN}
\subfile{capitoli/6_Esperimenti/6_4_2_StrutturaGRU.tex}


% Spiegazione delle codifiche utilizzate, con eventuale spiegazione del perchè non posso utilizzare per entrambe le reti la stessa codifica
% + eventuale codifica binaria ancora da testare
% Spiegazione delle tecniche utilizzate per la serializzazione dei modelli ed il perchè della scelte: pickle vs funzioni di libreria, menziono inoltre le differenze ottenute sopratutto utilizzando pytorch nelle dimensioni del modello prima e dopo l'addestramento
\section{Implementazione}
\label{sec:implementazione}
\subfile{capitoli/6_Esperimenti/6_5_Implementazione.tex}

\chapter{Risultati}
\label{chap:Risultati}
% Esperimenti al fine di valutare la bontà della rete ricorrente: voglio ottimizzare la rete ricorrente sul mio dataset e poi valutare le prestazioni sui filtri di bloom
% Esperimenti per valutare prestazioni della rete feedforward sul dataset + ottimizzazione dei parametri e del modello tramite funzioni di keras
\section{Valutazione dei classificatori}
\label{sec:valutazioneClassificatori}
\subfile{capitoli/7_Risultati/7_1_ValutazioneClassificatori.tex}


% Confronto risultati riguardanti le prestazioni dei classificatori e le relative differenze nelle prestazioni dei filtri appresi 
% Traggo le conclusioni degli esperimenti: è necessario utilizzare una rete complessa come quella ricorrente o posso utilizzare anche reti più "semplici" come una feed forward?
\section{Valutazione sull'impatto dei classificatori nei filtri di Bloom appresi}
\label{sec:confrontoFiltri}
\subfile{capitoli/7_Risultati/7_2_ConfrontoFiltri.tex}

\chapter*{Conclusioni}
\subfile{capitoli/8_Conclusioni/Conclusioni.tex}

\printbibliography
\end{document}